\documentclass[ignorenonframetext,]{beamer}
\setbeamertemplate{caption}[numbered]
\setbeamertemplate{caption label separator}{: }
\setbeamercolor{caption name}{fg=normal text.fg}
\beamertemplatenavigationsymbolsempty
\usepackage{lmodern}
\usepackage{amssymb,amsmath}
\usepackage{ifxetex,ifluatex}
\usepackage{fixltx2e} % provides \textsubscript
\ifnum 0\ifxetex 1\fi\ifluatex 1\fi=0 % if pdftex
  \usepackage[T1]{fontenc}
  \usepackage[utf8]{inputenc}
\else % if luatex or xelatex
  \ifxetex
    \usepackage{mathspec}
  \else
    \usepackage{fontspec}
  \fi
  \defaultfontfeatures{Ligatures=TeX,Scale=MatchLowercase}
\fi
% use upquote if available, for straight quotes in verbatim environments
\IfFileExists{upquote.sty}{\usepackage{upquote}}{}
% use microtype if available
\IfFileExists{microtype.sty}{%
\usepackage{microtype}
\UseMicrotypeSet[protrusion]{basicmath} % disable protrusion for tt fonts
}{}
\newif\ifbibliography
\hypersetup{
            pdftitle={Testing Scientific Code},
            pdfborder={0 0 0},
            breaklinks=true}
\urlstyle{same}  % don't use monospace font for urls

% Prevent slide breaks in the middle of a paragraph:
\widowpenalties 1 10000
\raggedbottom

\AtBeginPart{
  \let\insertpartnumber\relax
  \let\partname\relax
  \frame{\partpage}
}
\AtBeginSection{
  \ifbibliography
  \else
    \let\insertsectionnumber\relax
    \let\sectionname\relax
    \frame{\sectionpage}
  \fi
}
\AtBeginSubsection{
  \let\insertsubsectionnumber\relax
  \let\subsectionname\relax
  \frame{\subsectionpage}
}

\setlength{\parindent}{0pt}
\setlength{\parskip}{6pt plus 2pt minus 1pt}
\setlength{\emergencystretch}{3em}  % prevent overfull lines
\providecommand{\tightlist}{%
  \setlength{\itemsep}{0pt}\setlength{\parskip}{0pt}}
\setcounter{secnumdepth}{0}

\title{Testing Scientific Code}


\begin{document}
\frame{\titlepage}

\begin{frame}
\frametitle{What is a software test?}

\begin{itemize}
\item A sanity check that everything is working.
\item Guard against dangerous behaviour.
\item Checks inputs and outputs and makes sure answers are sensible.
\end{itemize}

\end{frame}

\begin{frame}
\frametitle{Scientific Method}
\begin{itemize}
\tightlist
\item
  Ask a question.
\item
  Form a hypothesis.
\item
  Make a prediction based on the hypothesis.
\item
  Design an experiment to test the hypothesis.
\item
  Build the equipment
\item
  \textbf{Calibrate equipment by testing it}
\item
  Run the experiment
\item
  Detail carefully the procedures and record information.
\item
  Draw conclusions from results.
\end{itemize}

\end{frame}

\begin{frame}
\framtitle{Scientific Method - Computational Science}

Do you check your `equipment' is calibrated? In many cases, the answer will be no.

Why?

\begin{itemize}
\item Ad-hoc changes are made to code which mean things stop working. 
\item Have you ever come back to a script that you wrote and found it no longer works because you changed a file somewhere
else?
\item Have These issues become especially common when multiple people, who
might not have oversight over what you are using it for, are working on
software. 

\emph{It's important to test the relevant parts of your code.}
\end{frame}

\begin{frame}{Principles of Software Testing}

\begin{itemize}
\tightlist
\item
  Software must meet the requirements set out (i.e. solve some problem)
\item  Software must respond correctly to different inputs.
\item  Output should be logically consistent with inputs.
\item  Software must run in a reasonable amount of time.
\item  Can be installed and run in the intended environment.
\end{itemize}

\end{frame}

\begin{frame}{Static vs Dynamic Testing}

\begin{block}{Static Testing}

\begin{itemize}
\tightlist
\item
  Reviews - Reviewing the code manually as a group - check for
  correctness.
\item
  Walthrough - Parties go through a products specification to give
  feedback on potential defects.
\item
  \textbf{Verification of design}
\end{itemize}

\end{block}

\begin{block}{Dynamic Testing}

\begin{itemize}
\tightlist
\item
  Executing portions of the code and checking the response.
\item
  \textbf{Validation of design}
\end{itemize}

\end{block}

\end{frame}

\begin{frame}{Why is it difficult for science?}

\begin{itemize}
\tightlist
\item
  Much of the time you don't know what the results will be.
\item
  Realistic problems can take time to run.
\item
  Testing requires computation of \emph{aggregate statistics}.
\item  If running simulations, then the behaviour should make sense over different timescales.
\item Simulations can be big and the results are not perfectly deterministic! Floating point error.
\end{itemize}

\end{frame}

\begin{frame}
\frametitle{Why is it crucial for science?}
\begin{itemize}
\tightlist
\item
  Often need to implement new algorithms or change things.
\item
  Changing without testing will break things.
\item
  You might need to come back to a previous experiment to use it as the
  basis for a new one; if the code is broken, you will have issues!
\end{itemize}
\end{frame}

\begin{frame}
\frametitle{Dynamic Testing: Types of Test}
\begin{block}{Unit Tests}
\begin{itemize}
\tightlist
\item
  Code is written in small, independent functions
\item
  Each function has tests that check that the result makes sense
\item
  For e.g.~a test of a function that calculates sin could check
  \(\sin{\left(n\pi\right)} = 0 \,\,\,\,\, \forall n \in \mathbb{Z}\)
\end{itemize}
\end{block}

\begin{block}{Integration Tests}
\begin{itemize}
\tightlist
\item
  Tests check that the system as a whole works convincingly.
\item For e.g.~for a simulation, a single integration test would check that
  the inputs to the simulation are correctly processed, the simulation
  proceeds without any errors, and that the results make sense.
\item Multiple integration tests are necessary for a large simulation
  package, such that all of the different potential combinations of
  algorithms are constructed.
\item Much, much harder to design, but essential - can be written so that
  they also serve as examples of how to use the software.
\end{itemize}
\end{block}

\begin{block}{Regression Tests}

\begin{itemize}
\tightlist
\item Check that results don't get less accurate over time.
\item Check that performance stays the same or gets better over time.
\end{itemize}

\end{block}

\end{frame}

\begin{frame}{Aside: Test Driven Development}

Methodology for testing where \textbf{tests are written before writing
any code}. This means....
\begin{itemize}
\tightlist
\item
  Before writing any code you need to plan out what the interface should
  be (i.e.~what arguments get passed to functions).
\item
  You must work out what the \emph{output} should be before writing any
  code.
\item
  Can form part of the 'agile' methodology for software development.
\item
  Can be difficult for developing scientific code where answers can be
  difficult to determine in advance.
\end{itemize}
\end{frame}

\end{document}
